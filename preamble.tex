%********** Graphics **********
\usepackage{xcolor}
\usepackage{graphicx}
\graphicspath{{graphics/}}
\usepackage{wrapfig}

%********** Miscellaneous **********
\usepackage{amsmath,amssymb,amsthm,array,mdwmath,mdwtab,fixltx2e}
\usepackage[hyphens]{url}
\usepackage{breqn}
\usepackage{subfigure}
\usepackage[round-precision=2,round-mode=figures,scientific-notation=true]{siunitx}
\sisetup{output-exponent-marker=\text{e}}
\usepackage[T1]{fontenc}

%********** PDF enhancements **********
\usepackage[pdftex,bookmarks=true,linktocpage]{hyperref}
\hypersetup{pdfauthor={Md Arefin and Raffi Khatchadourian},
	pdftitle={Porting the NetBeans Java 8 For Each Loop Lambda Expression Refactoring to Eclipse},
	pdfsubject={Software Engineering},
	pdfkeywords={automated,evolution,refactoring,integrated development environments,java}}
\hypersetup{colorlinks=true,citecolor=blue}
\usepackage{color}
\usepackage[pdftex]{thumbpdf}

%********** Enhanced lists **********
\usepackage{enumitem}

%********** Enhanced tables **********
\usepackage{booktabs}

%********** Natbib **********
\usepackage[numbers,square,sort&compress]{natbib} 
% \usepackage{natbibspacing}
% \let \cite = \citep
% \setcitestyle{aysep={}}
% \makeatletter
% \def\NAT@def@citea{\def\@citea{\NAT@separator}}
% \makeatother

%********** Saving paper **********
% \newcommand{\subparagraph}{}
% \usepackage{savetrees}
% \usepackage[small,it]{caption}
% \def\Section{\S}
\def\Section{Section~}
% \def\Figure{Fig.~}
\def\Figure{Figure~}
% \def\Example{Ex.~}
\def\Example{Example~}
% \def\Examples{\Example}
\def\Examples{Examples~}
% \renewcommand{\subsection}[1]{\textbf{\emph{#1.}}}
% \setlist{noitemsep,nosep}
% \usepackage[compact]{titlesec}

%********** Listings **********
\usepackage{listings}
\lstset{language=Java, 
    mathescape,
    tabsize=2, 
    numbers=none, 
    xleftmargin=0.2in,
    numberstyle=\tiny, 
    %basicstyle=\footnotesize\ttfamily,
    basicstyle=\small\sffamily,
    escapechar=~,
    commentstyle=\it,
    breaklines,
    breakatwhitespace=true,
    breakindent=2ex
}

%********** Algorithmic **********
\usepackage{algorithmic}
\algsetup{linenosize=\tiny,indent=.75em}

%---------------------------------------------------------------------- Physics

\newcommand{\Hamiltonian}{\mathcal{H}}
\newcommand{\Lagrangian}{\mathcal{L}}
\newcommand{\mean}[1]{\langle #1 \rangle}

%------------------------------------------------------------------------- Sets
\newcommand{\N}{\mathbb{N}}
\newcommand{\Z}{\mathbb{Z}}
\newcommand{\Q}{\mathbb{Q}}
\newcommand{\R}{\mathbb{R}}
\newcommand{\C}{\mathbb{C}}
\newcommand{\B}{\mathbb{B}}

\newcommand{\emtyset}{\varnothing}

\newcommand{\set}[1]{\ensuremath{\{#1\}}} 
\newcommand{\sequence}[1]{\ensuremath{\langle#1\rangle}} 
\newcommand{\Set}[1]{\ensuremath{\left\{#1\right\}}} 
\newcommand{\cardinality}[1]{\ensuremath{\vert#1\vert}} 
\newcommand{\suchthat}{\ensuremath{\bigm|}}
\newcommand{\union}{\cup}
\newcommand{\Union}{\bigcup}
\newcommand{\intersection}{\cap}
\newcommand{\Intersection}{\bigcap}
\newcommand{\powerset}[1]{\ensuremath{\mathscr{P}(#1)}}
\newcommand{\map}[3]{\ensuremath{#1\colon #2\to #3}}
\providecommand{\abs}[1]{\lvert#1\rvert}
\providecommand{\norm}[1]{\lVert#1\rVert}

%---------------------------------------------------------------------- Vectors
\renewcommand{\vec}[1]{\bm{\mathrm{#1}}}
\newcommand{\uvec}[1]{\vec{e}_{#1}}

\newcommand{\ket}[1]{\, |#1\rangle}
\newcommand{\bra}[1]{\langle~#1 |\,}

\newcommand{\rr}{\vec{r}}
\newcommand{\vv}{\vec{v}}
\newcommand{\aaa}{\vec{a}}

\newcommand{\zero}{\ensuremath{\mathbf{0}}}

%------------------------------------------------------------------ Derivatives
\newcommand{\D}[2]{\frac{d #1}{d #2}}
\newcommand{\DD}[2]{\frac{d^2 #1}{d #2^2}}
\newcommand{\PD}[2]{\frac{\partial~#1}{\partial~#2}}
\newcommand{\PDD}[2]{\frac{\partial^2 #1}{\partial~#2^2}}

%-------------------------------------------------------------------- Integrals
\newcommand{\oiint}{\int\hspace{-2ex}\int\hspace{-3.04ex}\bigcirc~}
\renewcommand{\iint}{\int\hspace{-1.5ex}\int}
\renewcommand{\iiint}{\int\hspace{-1.5ex}\int\hspace{-1.5ex}\int}

%-------------------------------------------------------------------- Proofs
%\newtheorem{dfn}{Definition}
\theoremstyle{remark}
% \theoremstyle{definition}
% \newtheorem{ex}{Example}[section]
\newtheorem{ex}{Example}
