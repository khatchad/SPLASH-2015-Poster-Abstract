\section{Introduction}

Java 8 is one of the most significant upgrades to Java programming language
and framework in over a decade. It provides supports for functional
programming, a new JavaScript engine, new APIs for date time manipulation, a
new stream API, and more~\cite{newjdk}. Such features can help make programs
easier to read, write, and maintain~\cite{gyori2013}.

Among the new Java 8 features, lambda expressions are touted to be most
significant. Lambda expression simplify the development process by
facilitating functional programming. They also provide a concise way to write
anonymous inner classes and make it easier to iterate through, filter, and
exact data from a \lstinline{Collection}~\cite{lambdaqs}.

Eclipse (\url{http://eclipse.org}) is one of the most popular IDEs for
Java. While Eclipse has incorporated several Java 8 feature
quick-fixes
%\footnote{Quick-fixes in Eclipse are content-assisted completions
%that help developers maintain and improve their code.} 
and refactorings, there
are still many features left to be done. For example, the NetBeans
IDE (\url{http://netbeans.org}) has a refactoring (originally proposed
by \citet{gyori2013}) that converts enhanced \lstinline{for} loops to a lambda
expression.  This paper discusses our ongoing work in exploring the porting of
such conversion mechanisms from NetBeans to the Eclipse IDE\@. 

Making such changes manually would require changing approx.~1,700 line of
non-commented, non-blank lines of code across approx.~100 files per project,
on average~\cite{gyori2013}. With our plug-in, Eclipse developers would not
need to make these changes manually.  Our tool will help Eclipse developers
adopt the functional-like programming model offered by Java 8. Furthermore, we
are in the process of exploring other Java 8 refactorings that do not
currently exist in any IDE\@. 
%More details can be found in Eclipse bug \#462725 (see
%\url{http://bugs.eclipse.org/bugs/show_bug.cgi?id=462725}).
